\section{Introduction}
Throughout the years, a lot of efforts were spent to study and derive the laws that govern the motion of a pendulum. \\
In many classical experiments, an observer measures directly the necessary physical quantities in order to estimate some other ones. \\
In our setting, we used an STM32F4Discovery (from now simply Discovery) board that, together with its sensors and an algorithm, it is able to compute directly and to provide to the observer the requested results while the experiment is performed. \\
The Discovery board is an electronic device equipped with an ARM CortexM4 microcontroller and various on-board devices. For this project, we used the LIS3DSH accelerometer that measures accelerations along the three axes, and a serial port that enables communication between the Discovery and a Personal Computer that eventually displays the results. \\

\subsection{Problem Statement}
A pendulum is a well studied physical system in which a weight is suspended from a pivot and it is able to oscillate freely. It is often difficult to find exact solutions in the general case so usually, some assumptions are made in order to make the problem easier to be modeled and solved. \\
In our case we model the setting as a simple gravity pendulum, i.e. assuming that:
\begin{itemize}
	\item the rod or cord on which the bob swings is massless, inextensible and always remains taut,
	\item the bob is a point mass,
	\item motion occurs only in two dimensions,
	\item the gravitational field is uniform,
	\item the friction can be neglected,
	\item the support does not move.
\end{itemize}
It is trivial to map the abstract case with our concrete one, in particular, the Discovery board is our bob and the USB cable is the rod. \\
The aim of the project is to estimate the length of the cable while the board oscillates.
\subsection{Summary of the Work}
Our work can be divided in three distinct phases:
\begin{itemize}
	\item we analyzed the datasheet \cite{accelerometerDatasheet} of on-board accelerometer and developed a driver that is able to interact with it while exposing a high-level interface that simplified interaction,
	\item we derived the mathematical model of the physical system in order to be able to separate between variables that we can measure and the unknown ones that we should compute,
	\item we drafted an algorithm to calculate the length of the cable, and evaluated its correctness using a simulator,
	\item we implemented the final algorithm setting some standard parameters and implementing complementary solutions as a result of various experiments. 
\end{itemize}
\pagebreak

\section{Design and Implementation}
We decided to implement the algorithm in a raw simulator built by us in Javascript \cite{javascriptPrototype}. Then we developed it in MIOSIX.\\
We will see the mathematical model, how to calculate the data we need ($\theta_{max}$ and period $T$) using raw data obtained by the accelerometer, how to filter and clean raw data and in which way we developed the driver for the accelerometer.

\subsection{Mathematical Model}
\label{sec:mathmodel}
We did not considered the friction (since it is to complex to calculate), so the period of the gravity pendulum has the following series as period formula \cite{pendulumSeries}: 
$$T = 2 \pi \sqrt{\frac{L}{g}} \left(1 + \frac{1}{16} \theta_0^2 + \frac{11}{3072} \theta_0^4 + ...\right)$$
Stopping at the second term of the series we are able to calculate the length of the cable ($L$) using period and $\theta_0$, i.e. $\theta_{max}$ , we obtain:
$$ L = g \cdot \frac{T^2}{4 \cdot \pi^2 \cdot \left(1 + \frac{\theta_0^2}{16}\right)^2}$$
\par
We can see that if we had used only one term of the series we would not have needed to calculate the $\theta_{max}$, but of course there is an error which increases for larger amplitude. In our application we cannot afford this error since, as we will see, in fact there are other imprecisions due to measurement errors and approximations, and the experiment is conducted using large amplitude.\\
We calculated the errors according to the values obtained using 3 terms of the series in table \prettyref{tab:errors}.  

\begin{table}[H]
	\small
	\begin{center}
		
		\caption{Series errors compared with the value obtained with 3 terms}
		\begin{tabular}{l|lcc|L{2cm}l}
			Angle & Error with 1 term & Error with 2 terms\\
			\cline{1-3}
			$0$ & 0\% & 0\%\\			
			$\frac{\pi}{4}$ & 4\% & 0.1\%\\			
			$\frac{\pi}{2}$ & 14.9\% & 1.8\%\\			
			$\pi$ & 49\% & 17.7\%\\			
		\end{tabular}
		\label{tab:errors}
	\end{center}
\end{table}

Since the pendulum is moved only by the ground acceleration and the cable is not rigid, the max angle of the pendulum is $\frac{\pi}{2}$ for that value the error using 2 terms is just of 1.8\% (an acceptable value). So 2 terms are enough, as used in the previous formula.

\subsection{Calculation of Data}
\label{sec:datacalc}

Since there is friction that is not considered in our mathematical model we have to calculate the period and $\theta_{max}$ in the shortest time possible, even if this means measurements errors. In fact a single measurement or a small measurement is not so good like the one obtained by a mean of measurements, but this error is smaller than the error caused by friction. We verified this thing in an experimental way.\par

We have considered only a quarter of period, in particular every descending quarter so we measure a period every new half period (this partially solves the problems due to friction).\\
We store the min value of accelerometer for y axis, that is easily converted in an angle using:
$$angle = \frac{\pi}{2} -\arcsin\left(\min\left\lbrace1.0, \left|\frac{y}{SCALE \cdot g}\right|\right\rbrace\right)$$
The min function is used to avoid domain problems due to errors of accelerometer that could give values higher than 1. We also used abs since the sign is not important and this simplifies everything.
Besides storing this we store also the timestamps of min value of y axis.\\
We know that the max value of y axis is when the pendulum is in vertical position, so when we see that the y is decrementing the pendulum is going up on the other side. So we have all data to calculate what we need.\par

The $\theta_{max}$ is given by the  angle previously calculated (angle for min value of y axis).\\
The period is given by the difference of timestamps, timestamp stored for min value of y axis and timestamp of vertical position (last position before decrementing).
Of course we recognize that we have completed the period in the next tick, when we are decrementing as explained previously. So we need to keep an history of the previous timestamp to be used in the calculation, in fact the previous one is the last of the period that we are going to calculate.

\subsection{Filter and Clean Raw Data}
After having done the first experiments we have seen a normal thing: the raw data obtained by accelerometer are not so good.\\
There are two main problems:
\begin{itemize}
	\item the accelerometer is not perfect and has an error
	\item the data have some pulses
\end{itemize} \par

Fortunately for the first problem the main errors are just shift errors and can be resolved internally by the accelerometer setting the right OFFSETS, so we needed just to find them.\\
To do that we put the Discovery in the two horizontal positions and we applied the following formula for each of both sides:
$$ offset = \frac{\left|1000-raw\_value\right|}{2}$$
then we have taken the mean of both values. \\
The values can have a little difference due to other errors not related to shift.\par

While for the second problem we decided to implement a low pass filter to remove the impulsive behavior of data. Implementing an exact low pass filter it is tedious and the equations that describe its behavior are computationally expensive to compute. We solved using an approximated model, and the formula \cite{Holt20045} that we used is:
$$ y_t = \alpha \cdot x_t + (1-\alpha) \cdot y_{t-1}$$
Where y is the value stored and x is the raw value.\\
Of course the higher alpha is, the more high frequency values are considered so the new value has more importance.\par

We have also to filter (exclude) small data (small angles and small periods obtained) as shown in~\ref{sec:results}in order  to avoid measuring non right movements like the ones obtained during the movement of the board by hands.

\subsection{LIS3DSH Driver}
We decided to implement the driver that is tightly integrated with MIOSIX, so to maximize reusability. In particular we inherited the base class \texttt{Device}, implementing its abstract methods. The driver registers itself to the \texttt{devfs} filesystem. This behavior is similar to the one that is commonly found in Linux. Thus from the client side, it is enough to know the name of the pseudo-file and open it much like one does for a canonical file in a mass-storage device.  \par The raw values acquired by the sensors are made available through the method \\ \texttt{SPILIS3DSHDriver::readBlock()} which is called by the \texttt{devfs} filesystem upon a read operation is requested by the client through the file handle. \\
The configuration of the sensor's parameters and the selection of relevant axis from which reads are later performed is carried out using the method \texttt{SPILIS3DSHDriver::ioctl()}. \\ We defined our set of commands and a \texttt{struct SPILIS3DSHDriver::Ioctl} that can be filled by the client when some sort of modifications in the driver/sensor state are required, or it is filled by the method itself when a client requests the current state of the driver/sensor. \\ With this approach we emphasize the fact that albeit simple, it gives enough flexibility in terms of API and extendability.
\par
The physical protocol used for communicating with the sensor is Serial Peripheral Interface \cite{accelerometerDatasheet}\cite{STM32F4RefMan}  (SPI). It is one of the most widely used for interfacing a microcontroller with external devices and other microcontrollers. \\  
Initialization of the serial link is carried out in three phases:
\begin{itemize}
	\item GPIO pins configuration, in order to utilize this protocol, $3 + n$ physical connections: \texttt{SCLK} for clock that synchronizes all peripherals, \texttt{MISO} for data coming from slave devices, \texttt{MOSI} for data going to slave devices, and $n$ \texttt{CS} pins, that each one of them signals to the matching slave peripheral that it has acquired the bus and can transfer data using it. In the Discovery board these pins are actually GPIO ones \cite{STM32F4RefMan}, that are configured for alternating functions and are permanently consumed as long as SPI is utilized,
	\item SPI Master initialization, SPI can be used by design in a shared environment by means of a bus. Before any actual exchange of data can be performed, the master component, which carries out the arbitration of such bus, must be properly initialized,
	\item actual sensor initialization, the driver checks that the sensor is actually present, and correctly replies to the commands. By default, all three axes are enabled and the output data rate is set to \SI{400}{\Hz}. Moreover, the sensor has a configurable range scale for measuring accelerations. Possible intervals are: $\pm$\SI{2}{\g}, $\pm$\SI{4}{\g}, $\pm$\SI{6}{\g}, $\pm$\SI{8}{\g} and $\pm$\SI{16}{\g}. In order to get the maximum precision achievable by the sensor, we used the scale ranging from \SI{-2}{\g} to \SI{+2}{\g}.
\end{itemize}
\pagebreak

\section{Experimental Evaluation}
We conducted several experiments with different hypotheses and values for the parameters defined in the previous section. \\
We followed a step by step approach in order to test our models and procedures. This permitted us to validate our solutions and quickly assess the correctness of our model. \\
As explained shortly, during this phase we faced some issues that we had to solve in order to derive the requested results. \\
We will demonstrate that our design and implementation worked fairly well in experiments.
\subsection{Experimental Setup}
\subsubsection{Evaluating the Accelerometer Performance}
In this step we checked that both driver and accelerometer worked well. As already explained we discovered that adjustments were needed in order to be able to use the acquired data for the next step. In particular we determined the correct amount of offset that we need to add to the raw values measured by the accelerometer. \par It is important to notice that although the sensor is the same for all Discovery boards, every instance has small differences between other ones so this means that our value might not work well for other boards thus this kind of calibration must be done every time a new Discovery is used for experiments. \\ Besides this, no particular issues were found, the calibration procedure is simple enough and we are quite confident about the value we determined.
\subsubsection{Measuring the Angle}
Given the data provided by the accelerometer and the formula to compute the angle, we are now able to verify if these values match the physical situation. During this step, we became aware by the fact that using a PC to power up the Discovery is not too much comfortable so we used an external power source: a simple group of batteries should be enough, in our case we used a Powerbank since it already has the capability to distribute power by means of an USB port. It turned out that this solution worked extremely well also in the next steps so, from now on, we will use this method while running all the experiments. \par The angle is measured w.r.t the vertical position, i.e. from the point of stable equilibrium of the pendulum. Apart from this issue, no other ones were encountered: no particular tools were used in the real world to measure the angle, only a visive check, but again, given the simplicity of the experiment, we could assure that the values computed by the Discovery are correct. 
\subsubsection{Measuring the Period}
At this step, the experiment more or less resembles the final one, except for the fact that values displayed on the computer is the angle $\theta_{max}$ coupled with the period. \\
It is here that we found out a couple of issues that we had to solve:
\begin{itemize}
	\item \textit{Serial communication}, the Discovery uses some pins attached to it for implementing the serial port (not the USB port itself), so another cable must be used in order to be able to display the values on the PC while running the experiments. This poses a serious problem: this new cable introduces much resistance that completely prevents the Discovery from oscillating! We solved the issue by substituting the cable with a ZigBee board. This device, coupled with a receiver attached to the PC through an USB port, uses a radio frequency for transmitting data using the serial interface, thus with little effort and no need to resort to some fancy communication protocols we are able to send a stream of data via wireless while still using the old and simple serial interface.
	\item \textit{Pulses}, the main loop of the algorithm reads continuously fresh data from the sensors, since we are living in a real world, and although we modeled our problem with only one dimensional variable, sometimes, during oscillations, the Discovery rotates briefly on its other two axes. This situation, amplified by the impulsive behavior of data, showed that the measurements of the period were incorrect. As already explained, by using a low pass filter on top of data acquired by accelerometer we are able to significantly boost the performance of our system up to the point where values become correct enough for a good estimation of the cable length. 
\end{itemize}
With these forethoughts, the experiment simply consists of letting the Discovery oscillate freely after releasing it starting from a particular angle. We noticed that some delay between every iteration of the main loop must be introduced since at beginning the ZigBees must synchronize each other in order to establish a correct communication. If we send data with a very high ratio the internal buffer fills up and some strange things might happen, for instance some random lags appear, moreover this situation might still occur even after proper synchronization.\par
Moreover we used this phase to check some assumptions done in the design such as the error caused by friction as explained in section~\ref{sec:datacalc}.
\subsubsection{Estimating the Length of the Cable}
We are able now to estimate the required unknown variable. \\ We applied the formula explained before and coupled with another low pass filter, the length of the cable is displayed on the PC. Using some measuring tool, we assessed the real length of the cable and compared it with our estimated value. We found out that although the majority of the output is correct, seldom some strange and incorrect values are displayed. \par This is not so easy to justify, one possible reason is that the low pass filters are actually an approximation and for some ill-conditioned situations they simply fail to provide meaningful results. Another reason is that sometimes there are some impulses of data that imply that the algorithm does not work properly (it can stop itself in advance).\\
A possible future scenario might involve a more in depth study of these problems and possibly how to solve them as we will quickly see in section~\ref{sec:conclusions}.
\subsection{Results}
\label{sec:results}
Given the previous set of experiments we derived the following results:
\begin{itemize}
	\item a good value for offset is $offset_y = 10$,
	\item we assessed the correct estimation of well known angles like $30\si{\degree}$, $45\si{\degree}$, $60\si{\degree}$ and $90\si{\degree}$. We did not evaluated angles $> 90\si{\degree}$, since they are not feasible with our model (as explained previously in section~\ref{sec:mathmodel}),
	\item for $\alpha_y$, we evaluated different possible values in the interval $0.5 \leq \alpha_y < 1 $. That lower bound is justified by the fact that we are more interested in new values than in old ones, while still considering by some extent the history. We found that a good choice is $\alpha_y = 0.5$ that gives equal importance to both old and new values,
	\item a similar reasoning can be done for $\alpha_{length}$, we again restricted the search space in the same interval as before, this time determining that a good value is $\alpha_{length} = 0.8$, so we are more interested in new values than in old ones,
\end{itemize}
It is important to notice that these experiments were conducted with angles $\gtrsim 30\si{\degree}$, for multiple reasons. For instance we noticed that the estimation becomes more correct while the experiment proceeds. With low angles the duration is not long enough in order to come up with good results. Moreover we filtered low angles to avoid making measurements during the moving of the Discovery.
With these hypotheses we are able to correctly estimate, in the majority of the cases, the cable length with a precision of less than \SI{1}{\cm}.
\pagebreak

\section{Conclusions and Future Works}
\label{sec:conclusions}
This project was very interesting because it joins two different topics (embedded systems and physics), but both engineering sciences. So it is very useful to understand the different kinds of applications of a "simple" thing like an accelerometer.\\
Doing this project was not so difficult, because we have made some reasonable assumptions to simplify the problem, a typical approach of engineering problems. Of course our application was not perfect and in some cases we obtained wrong data, but as we will see now these things can be fixed.\par
There are some different things that can be implemented to improve the application, we can divide them in two main groups:
\begin{itemize}
	\item increase precision
	\item avoid errors
\end{itemize}
To increase precision there are some improvements that we can do, such as: using more terms in the series, considering the friction, making a mean of value obtained (for example for period) and so on.\\
While to avoid errors we can introduce some logics to detect wrong behaviors and consequentially wrong data, common situations are:
\begin{itemize}
	\item board moved by hand
	\item the cable, since it is not a perfect rope, can create wrong movements and also some big rotations (small are negligible)
	\item with some rotations the Discovery could receive some pulses less than the previous values, that can stop the actual algorithm (since it stops itself when there is a reduction of values as explained in section~\ref{sec:datacalc}).
	\item board can receive a force at the beginning of the oscillation that makes first data not real
\end{itemize}
Some of them are easy to be solved or it is easy to find a workaround. For example to solve in part the last problem (external force) we can simply see if the accelerometer resultant is the same as g.

